%Este trabalho está licenciado sob a Licença Creative Commons Atribuição-CompartilhaIgual 3.0 Não Adaptada. Para ver uma cópia desta licença, visite https://creativecommons.org/licenses/by-sa/3.0/ ou envie uma carta para Creative Commons, PO Box 1866, Mountain View, CA 94042, USA.

\chapter{Campos vetoriais}\label{cap:campos}\index{campos vetoriais}


\chapter{Campos escalares e campos vetoriais}
 Campo é termo usado para designar funções definidas em uma porção do espaço tridimensional (ou bidimensional), isto é, funções cujo domínio $D$ é um subconjunto de $\mathbb{R}^3$ (ou $\mathbb{R}^2$). Trabalharemos com dois tipos de campos: os campos escalares e os campos vetoriais. Os campos vetoriais são funções cuja imagem é composta de vetores no $\mathbb{R}^3$, já a imagem dos campos escalares são números reais, isto é, escalares.
 
\begin{ex}\label{excampos} São exemplos de campos escalares.
\begin{itemize}
\item [a)] A função que liga a posição de um ponto dentro de uma sala à temperatura neste ponto.
\item [b)] A pressão do ar como função da posição na atmosfera.  
\item [c)] $f(x,y,z)= 100 + 20e^{-\sqrt{x^2+y^2+z^2}}$.
\item [d)] $f(x,y,z)= \vec{r} \cdot \vec{r} = x^2+y^2+z^2$, onde $\vec{r}=x\vec{i}+y\vec{j}+z\vec{k}$ .
\end{itemize}
\end{ex}   


\begin{ex}\label{excampos} São exemplos de campos vetoriais.
\begin{itemize}
\item [a)] A função que liga a posição de um ponto dentro de uma fluido à velocidade (vetor) neste ponto.
\item [b)] O campo magnéticos, elétrico, gravitacional etc.  
\item [c)] $\vec{F}(x,y,z)= x\vec{i} + z\vec{j} - y\vec{k}$.
\item [d)] $\vec{F}(x,y,z)= \vec{r}\times \vec{k}$
\end{itemize}
\end{ex}  

\section{Representação gráfica dos campos vetoriais}
Um campo vetorial é representado graficamente por um conjunto de setas partindo de pontos $(x,y,z)$ e de comprimento proporcional ao módulo de $\vec{F}(x,y,z)$ e mesma direção e sentido de $\vec{F}(x,y,z)$. O conjunto de pontos é escolhido de forma arbitrária de forma a permitir interpretar o campo.  

\begin{ex} Represente graficamente o campo vetorial $\vec{F}(x,y)=\sqrt{y}\ \!\vec{i},~~y\geq 0$.
%\begin{figure}[htp]
\includegraphics{cap_campos/figs/campo_exemplo_1}
%\caption{\label{campo_radial}Representação gráfica do campo $\vec{F}(x,y)=\sqrt{y}\ \!\vec{i},~~y\geq 0$.}
%\end{figure}
\end{ex}


\begin{ex} Represente graficamente o campo vetorial $\vec{F}(x,y)=x\vec{i},~~y\geq 0$.
%\begin{figure}[htp]
\includegraphics{cap_campos/figs/campo_exemplo_2}
%\caption{\label{campo_radial}Representação gráfica do campo $\vec{F}(x,y)=x\vec{i}$.}
%\end{figure}
\end{ex}

\begin{ex} Represente graficamente o campo vetorial $\vec{F}(x,y)=-y\vec{i}+x\vec{j}$.
%\begin{figure}[htp]
\includegraphics{cap_campos/figs/campo_exemplo_3}
%\caption{\label{campo_radial}Representação gráfica do campo $\vec{F}(x,y)=-y\vec{i}+x\vec{j}$.}
%\end{figure}
\end{ex}


\section{Campos radiais}
Campos radiais vetoriais \index{campos radiais} são campos da forma $\vec{F}=f(r) \hat{r}$, isto é campos vetoriais cujo módulo depende apenas da distância até a origem, isto é, de $r=\|\vec{r}\|=\sqrt{x^2+y^2+z^2}$ e cuja direção é sempre paralela ao vetor posição, $\vec{r}$.
\begin{ex} Represente graficamente o campo vetorial $\vec{F}=\vec{r}$ no plano $xy$.
%\begin{figure}[htp]
\includegraphics{cap_campos/figs/campo_radial}
%\caption{\label{campo_radial}Representação gráfica do campo $\vec{F}=\vec{r}$}
%\end{figure}
\end{ex}



\section{Tipos de campos vetoriais}

\construirSec

\subsection*{Exercícios resolvidos}

\construirExeresol

\begin{exeresol}
  Um exercício.
\end{exeresol}
\begin{resol}
  Resolução do exercício.
\end{resol}

\subsection*{Exercícios}

\construirExer

\begin{exer}
  Um exercício.
\end{exer}
\begin{resp}
  Resposta curta do exercício.
\end{resp}

\section{Exercícios finais}

\construirExer

\begin{exer}
  Um exercício.
\end{exer}
\begin{resp}
  Resposta curta do exercício.
\end{resp}

