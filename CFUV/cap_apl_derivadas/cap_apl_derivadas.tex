%Este trabalho está licenciado sob a Licença Creative Commons Atribuição-CompartilhaIgual 3.0 Não Adaptada. Para ver uma cópia desta licença, visite https://creativecommons.org/licenses/by-sa/3.0/ ou envie uma carta para Creative Commons, PO Box 1866, Mountain View, CA 94042, USA.

\chapter{Aplicações das derivadas}\label{cap:apl_derivadas}\index{aplicações! derivadas}

\emconstrucao

A ideia deste capítulo é desenvolver algumas aplicações a partir do que foi construída anteriormente. Além disso, esta seção também busca fornecer ao leitor uma justificativa teórica do porquê as derivadas são importantes. 

	É interessante sempre ter em mente que a derivada de uma função $f$ em um ponto é uma propriedade local, i.e., que determina o comportamento da função ``perto'' do ponto em questão. Contudo, em uma primeira viagem, pode não ser exatamente nítido de que maneira podemos relacionar uma propriedade local (a derivada da $f$) com uma global (o comportamento da $f$). Esta relação entre o global e o local ficará mais evidente ao decorrer do texto, principalmente quando for abordado o Teorema Fundamental do Cálculo. Entretanto, este capítulo já é um bom começo para entender a importância do conceito de derivada e como ela pode ajudar no estudo de funções.
	
	Antes de realmente mergulhar nas contas, é importante destacar que seria interessante o leitor manter sempre consigo a ideia que deu luz ao conceito de derivada: a reta tangente. Ter isto em mente a todo o momento facilita a visualização e ajuda a criar uma intuição sobre o assunto. Abaixo, provaremos uma série de propriedades que, ao enunciadas, realmente ``parecem'' ser verdadeiras. Porém, ``parecem'' apenas sobre a perspectiva de quem realmente domina o alicerce fundamental sobre o qual a derivada foi construída. E desenvolver essa intuição é essencial para a compreensão do assunto como um todo.
 
\section{Teorema do valor médio}\index{teorema!do valor médio}

\construirSec

	Vamos primeiramente enunciar o teorema:
	\begin{teo}[Teorema do valor médio]
		Seja $f:[a,b]\to \mathbb{R}$ tal que $f$ é contínua em $[a,b]$ e derivável em $(a,b)$, então existe $c \in (a,b)$ tal que:
		$$\frac{f(b) - f(a)}{b - a} = f'(c)$$
	\end{teo}
	Antes de passar diretamente para a prova do teorema, é interessante que o leitor entenda o sentido da equação a acima. Observe que o lado direito representa a taxa de variação instantânea em um ponto, enquanto o lado esquerdo, a taxa de variação média no intervalo. Logo, dentro das hipóteses do teorema, estamos afirmando que a taxa de variação média coincide com a instantânea em pelo menos um ponto do intervalo. 
	
	Para entender a importância disto, vamos estudar primeiramente um caso discreto: imagine que queremos calcular a média aritmética de uma certa quantidade de números reais positivos. Nesta situação, é possível que nenhum valor da amostra coincida com o valor da média. Um exemplo simples onde isso acontece é com os números $8,10$, que possuem média $9$. O problema aqui é que (dentre possíveis outras coisas) estamos ``pulando'' valores, justamente por estarmos trabalhando em um contexto discreto. No caso contínuo, isso não acontece. Sabemos que dada uma função contínua e dois elementos de sua imagem, ela sempre assume todos os valores entre esses elementos (teorema do valor intermediário).
	
	Veremos que atacar este problema utilizando a continuidade da função será o caminho que nos permitirá demonstrar o teorema. A maneira mais comum de prová-lo é utilizando um teorema auxiliar, chamado teorema de Rolle, que nada mais é que um caso especial do nosso objetivo principal.
	\begin{teo}[Teorema de Rolle]
		Seja $f:[a,b]\to \mathbb{R}$ tal que $f$ é contínua em $[a,b]$ e derivável em $(a,b)$, com $f(a) = f(b)$. Então existe $c \in (a,b)$ tal que:
		$$f'(c) = 0$$
	\end{teo}
	\begin{proof}[Prova do teorema de Rolle]
		Como $f$ é contínua, podemos separar a demonstração em 3 casos:
		\begin{description}
			\item[Caso 1:] $f$ assume o seu mínimo em $(a,b)$:
		
			Digamos que $f$ assuma seu mínimo em $c \in (a,b)$, logo $c$ é ponto crítico de $f$. Portanto, $f'(c)=0$.
			\item[Caso 2:] $f$ assume o seu máximo em $(a,b)$:
			
			Analogamente, suponha que $f$ assuma seu máximo em $c \in (a,b)$, logo $c$ é ponto crítico de $f$ e, da mesma maneira, temos $f'(c)=0$.
			\item[Caso 3:] $f$ não assume o seu máximo nem mínimo em $(a,b)$:
			
			Neste caso, observe que $f$ deve assumir seu máximo em algum dos pontos $a$ ou $b$, assim como também deve assumir seu mínimo nesses pontos. Como $f(a) = f(b)$, temos que o máximo e o mínimo de $f$ coincidem em $[a,b]$, logo $f$ é constante em $[a,b]$, ou seja $f'(x)=0$ para todo $x \in (a,b)$.
			
			Em cada um dos casos, garantimos a existência de um $c\in(a,b)$ tal que $f'(c) =0$, o que conclui a demonstração.
		\end{description}
	\end{proof}
	
	Agora, com esta ferramenta, podemos prosseguir com a demonstração de um dos teoremas mais importantes dessa seção:
	
	\begin{proof}[Prova do teorema do valor médio]
		Vamos definir $g:[a,b]\to\mathbb{R}$, tal que:
		$$g(x) = f(x) - \frac{f(b) - f(a)}{b-a}(x-a)$$
		observe que $g$ é contínua em $[a,b]$ e derivável em $(a,b)$, pois é uma combinação linear de funções com estas propriedades. Ainda:
		$$g(a)= f(a) - \frac{f(b) - f(a)}{b-a}(a-a) = f(a)$$
		$$g(b)= f(b) - \frac{f(b) - f(a)}{b-a}(b-a) = f(b) - (f(b)-f(a)) = f(a)$$
		
		Logo $g(a)=g(b)$. Portanto, pelo teorema de Rolle, temos que existe $c\in(a,b)$, tal que $g'(c)=0$. Agora, observe que:
		
		$$g'(x) = f'(x) - \frac{f(b) - f(a)}{b-a}$$
		Logo, temos:
		$$g'(c)= 0 = f'(c) - \frac{f(b) - f(a)}{b-a} \iff f'(c) = \frac{f(b) - f(a)}{b-a}$$
		Com isso, obtivemos $c\in (a,b)$, tal que $f'(c) = \frac{f(b) - f(a)}{b-a}$ e isto conclui a demonstração
	\end{proof}
	
	Finalizamos a discussão sobre o Teorema do valor médio com uma aplicação importante deste resultado.
	\begin{teo}
		Seja $f:[a,b]\to \mathbb{R}$ tal que $f$ é contínua em $[a,b]$ e derivável em $(a,b)$. Se $f'(x)=0$ para todo $x\in(a,b)$, então existe $A\in\mathbb{R}$, tal que $f(x)=A$ para todo $x\in[a,b]$. 
	\end{teo}
	Em outras palavras, se a taxa de variação de uma função é nula, então ela não varia (obvio!).
	
	\begin{proof}[Prova]
		Vamos mostrar que $f(x) = f(a) = A$ para todo $x\in[a,b]$.:
		
		 Se $x=a$, a igualdade é trivial. Caso $x\in(a,b]$, então pelo teorema do valor médio, existe $c\in(a,x)$ tal que:
		 $$\frac{f(x) - f(a)}{x-a} = f'(c) = 0$$
		 Logo, $f(x) = f(a) = A$ para $x\in(a,b]$. Com isso, temos $f(x)= A$ para todo $x\in[a,b]$, finalizando a demonstração.
		
	\end{proof}
	
	\begin{cor}
		Sejam $f,g:[a,b]\to \mathbb{R}$ tal que $f,g$ são contínuas em $[a,b]$ e deriváveis em $(a,b)$. Se $f'(x)= g'(x)$ para todo $x\in(a,b)$, então existe $A\in\mathbb{R}$, tal que $f(x) = g(x)+ A$ para todo $x\in[a,b]$.
	\end{cor}
	\begin{proof}
		Basta tomar $h(x) = f(x)-g(x)$ e aplicar o teorema anterior.
	\end{proof}

\subsection*{Exercícios resolvidos}

\construirExeresol


\subsection*{Exercícios}

\construirExer


%***********************************************************************************
\section{Funções crescentes e decrescentes}\index{funções!crescentes}\index{funções!decrescentes}\index{funções!monótonas}
\construirSec

\subsection*{Exercícios resolvidos}

\construirExeresol


\subsection*{Exercícios}

\construirExer


%***********************************************************************************
\section{Máximos e mínimos}\index{máximos}\index{mínimos}\index{pontos!de máximo}\index{pontos!de mínimo}
\construirSec

\subsection*{Exercícios resolvidos}

\construirExeresol


\subsection*{Exercícios}

\construirExer



%***********************************************************************************
\section{Concavidade}\index{concavidade}
\construirSec

\subsection*{Exercícios resolvidos}

\construirExeresol


\subsection*{Exercícios}

\construirExer


%***********************************************************************************
\section{Problemas de otimização}\index{otimização}\index{problemas!de otimização}
\construirSec

\subsection*{Exercícios resolvidos}

\construirExeresol


\subsection*{Exercícios}

\construirExer


%************************************************************************************
\section{Exercícios finais}

\construirExer


