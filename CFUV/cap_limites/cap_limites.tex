%Este trabalho está licenciado sob a Licença Creative Commons Atribuição-CompartilhaIgual 3.0 Não Adaptada. Para ver uma cópia desta licença, visite https://creativecommons.org/licenses/by-sa/3.0/ ou envie uma carta para Creative Commons, PO Box 1866, Mountain View, CA 94042, USA.

\chapter{Limites}\label{cap:limites}\index{limite}

\emconstrucao

\section{A ideia de limite}\label{sec:limites_ideia}
\subsection {Limites no infinito}
\construirSec

\section{Operações com limites}\label{sec:limites_opera}\index{limites!operações}
\construirSec

\section{Funções contínuas}\label{sec:limites_fun_cont}\index{funções!contínuas}\index{continuidade}
\construirSec

\section{Funções exponenciais e logaritmicas}\label{sec:limites_exp_log}\index{exponenciais}\index{logaritmos}\index{funções!exponenciais}\index{funções!logartimicas}
\construirSec

\section{Funções trigonométricas}\label{sec:limites_fun_trigo}\index{funções!trigonométricas} %Limites fundamentais envolvendo funções trigonométricas
\construirSec

\section{Definição rigorosa de limite*}\label{sec:limites_def_rigor}
\construirSec

\subsection*{Exercícios resolvidos}

\construirExeresol

\begin{exeresol}
  Calcule $\lim_{x\to 2} 2x$.
\end{exeresol}
\begin{resol}
  Podemos calcular o limite solicitado da seguinte forma
  \begin{equation}
    \begin{split}
      \lim_{x\to 2} 2x &= 2\lim_{x\to 2} x\\
      &= 2\times 2 = 4.
    \end{split}
  \end{equation}
\end{resol}

\subsection*{Exercícios}

\construirExer

\begin{exer}
  Calcule $\lim_{x\to 3} 5x$.
\end{exer}
\begin{resp}
  15
\end{resp}

\section{Exercícios finais}

\construirExer

\begin{exer}
  Calcule $\displaystyle\lim_{x\to \infty} \frac{e^{-x}}{x^{99999}}$.
\end{exer}
\begin{resp}
  0
\end{resp}

